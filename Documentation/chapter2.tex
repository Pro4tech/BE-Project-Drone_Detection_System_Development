%DO NOT MESS AROUND WITH THE CODE ON THIS PAGE UNLESS YOU %REALLY KNOW WHAT YOU ARE DOING
\chapter{Literature Survey} \label{Literature Survey}
\section{Introduction} \label{Introduction}
An extensive literature survey is done on Drone Frequency arrange and their Power Levels to Construct a detection logic and use these methods of detection based on the requirement along with Drone Regulations in India


This paper \cite{1} gives us a broader understanding of the process of signal capturing and dataset creation Using RF Signature of drone Signals. In the paper they have generated a dataset of 3 drone manufacturers(AR Drone,Phantom Drone and Bebop Drone)


This whitepaper \cite{2} contains the policy roadmap  which gives a better idea regarding  the rules and regulation imposed by the Indian government under Civil Aviation Regulations (CAR) for UAS for commercial and non-commercial drone users.  It also provided us with a better understanding of the  rules and regulations with their respective constraints on the UAS and the Pilot. It further shows the importance of training for the pilot in flying the drone


This website \cite{4} provides the instructions for initial setup and configuration of PlutoSDR in a virtual machine and introduction to digital signal processing in python

\cite{5} gave us a broad idea about the types of ambient signals present in the Wi-fi frequency range. It also provided us with a better understanding of the Wi-Fi signal.

\cite{6} helped us to differentiate between and categorize the drone, Wi-Fi and Bluetooth signals using their RF Fingerprints. It also gave an idea of how to implement machine learning for signal classification using different methods.


%\noindent 1)  Every chapter has to start on a new page;\\
%\noindent 2)  Min.6 and Max.7 Chapters; \\
%\noindent 3)  Chapter title and contents of chapter 4 and 5 can vary depending on type of project(hardware/software/research(M. E.))\\
%\noindent 4)  Chapter 1 to 7 = No. of pages\\
%\noindent 5)  Min. number of pages 40; \\
%\noindent 6)  Header and Footer is a must. (Header: TITLE OF THE PROJECT, Footer: NAME OF THE DEPARTMENT, GEC - 2019)\\
%\noindent 7)  Font : Times New Roman \\
%\noindent 8)  Font Size 12\\
%\noindent 9)  Spacing between lines: 1.5;\\
%\noindent 10)  Margin 1.25 all sides \\
%\noindent 11) References as per IEEE format. Students are requested to refer to the following link to make sure the citations/ references are according to IEEE Format:\\https://ieee-dataport.org/sites/default/files/analysis/27/IEEE%20Citation%20Guidelines.pdf

